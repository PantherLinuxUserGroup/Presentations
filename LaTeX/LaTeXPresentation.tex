\documentclass{beamer}
\usepackage{hyperref}
\usepackage{listings}

\usetheme{Berlin}


\title {{\LaTeX}}
\subtitle{Creating professional looking 
          documents and presentations}
\author{Erik Edrosa}
\institute [PLUG]{Panther Linux User Group}
\subject{Presentation}
\date{}




\begin{document}

\lstset{language=[LaTeX]TeX}
  \frame{\titlepage}
  \begin{frame}
  \frametitle{What is \TeX?}
  \framesubtitle{pronounced tek}
  %content goes here
  \begin{itemize}
    \item a typesetting markup language
    \item created by Donald Knuth
    \begin{itemize}
      \item A print of one of the volumes of The Art of Computer Programming was awful
      \item because of the change from hot metal typesetting to photographic techniques losing the "good classic style"
      \item intended to be compatible on all computers
    \end{itemize}
    \item great for typesetting complex mathimatical formula
  \end{itemize}
  \[ x = \frac{-b\pm \sqrt{b^2 - 4ac}}{2a} \]
  \[\lim_{x \to \infty} \exp(-x) = 0 \]
  \end{frame}
  \begin{frame}
    \frametitle{What is {\LaTeX}}
    \framesubtitle{pronounced Lay-tek or Lah-tek}
    \begin{itemize}
      \item a collection of \TeX macros
      \item allows the writer to focus on the content and not the appearance
      \item can be extended by packages
      \item used widely by academia for creating documents and presentations
    \end{itemize}
    %More content goes here
  \end{frame}
  \begin{frame}
    \frametitle{Getting {\LaTeX}}
    \begin{itemize}
      \item Linux
        \begin{itemize}
          \item \TeX Live \url{http://www.tug.org/texlive/}
          \item Check your distro for a texlive package or install from source
        \end{itemize}
      \item Windows
        \begin{itemize}
          \item MiK{\TeX}  \url{http://www.miktex.org/}
        \end{itemize}
      \item Mac OS
        \begin{itemize}
          \item Mac{\TeX} \url{http://tug.org/mactex/}
        \end{itemize}
      \item Online editors
        \begin{itemize}
          \item {\LaTeX} Lab \url{http://docs.latexlab.org/}
          \item Share{\LaTeX} \url{https://www.sharelatex.com/}
        \end{itemize}
    \end{itemize}  
  \end{frame}
  \begin{frame}[fragile]
    \frametitle{Basics of {\LaTeX}}
    \framesubtitle{Simple document}
    \begin{lstlisting}[frame=single]
% firstDocument.tex
\documentclass{article} 

\begin{document} % Start actual document
Hello World!
\end{document} % finish document
    \end{lstlisting}
    To compile to a pdf just type pdflatex firstDocument.tex
\end{frame}  
\begin{frame}[fragile]
  \frametitle{The documentclass}
  \begin{itemize}
      \begin{lstlisting}
\documentclass{article}
      \end{lstlisting}
    \item documentclass determines what kind of document
\begin{itemize}
  \item article - most general document type
  \item book - for writing real books
  \item letter - for writing letters
  \item beamer - for creating presentations
  \item many, many more
\end{itemize}
  \end{itemize}
\end{frame}
\begin{frame}[fragile]
  \frametitle{The document body}
  \begin{lstlisting}
\begin{document}
\end{document}
  \end{lstlisting}
  \begin{itemize}
    \item enclosed within begin and end
    \item the main content of your document goes here
  \end{itemize}
\end{frame}
\begin{frame}[fragile]
  \frametitle{Title pages}
  \framesubtitle{example title mark up}
  \lstinputlisting[firstline=7, lastline=14, basicstyle=\scriptsize]{LaTeXPresentation.tex}
  \begin{itemize}
    \item just use \textbackslash{}titlepage in the document
  \end{itemize}
\end{frame}
\begin{frame}[fragile]
  \frametitle{itemize}
  \framesubtitle{creating lists}
  \lstinputlisting[firstline=94, lastline=100]{LaTeXPresentation.tex}
  \begin{itemize}
    \item create lists with itemize and each bullet will be an item
    \item can even nest lists inside other lists
  \end{itemize}
\end{frame}
\begin{frame}[fragile]
  \frametitle{creating sections}
  \begin{lstlisting}
\section{main header}
% section content here
\subsection{subheader}
% subsection content here
\subsubsection{subsubheader}
% subsubsection content here
\end{lstlisting}
\begin{itemize}
  \item define sections in your document
  \item can even include automatic numbering
  \item \textbackslash{}tableofcontents generates a table of contents with all sections
\end{itemize}
\end{frame}
\begin{frame}[fragile]
  \frametitle{Creating a presentation with beamer}
  \framesubtitle{presentation template}
  \lstinputlisting[basicstyle=\tiny]{presentation.tex}
\end{frame}
\begin{frame}
  \frametitle{Useful resources}
  \begin{itemize}
    \item \href{http://latex-project.org/}{{\LaTeX} project website}
    \item \href{http://ctan.org/}{CTAN (Comprehensive {\TeX} Archive Network)}
    \item \href{http://en.wikibooks.org/wiki/LaTeX}{{\LaTeX} wikibook}
    \item \href{https://www.sharelatex.com/templates/}{ShareLaTeX templates}
  \end{itemize}
\end{frame}
\begin{frame}
  \frametitle{Thank you}
  \begin{itemize}
    \item Questions or comments?
    \item Contact me at eedro001@fiu.edu
    \item visit the club's website
      \begin{itemize}
        \item \url{http://plug.cs.fiu.edu}
      \end{itemize}
    \item Join our irc channel
      \begin{itemize}
        \item plug.cs.fiu.edu
        \item room \#chat
      \end{itemize}
  \end{itemize}
\end{frame}

\end{document}
